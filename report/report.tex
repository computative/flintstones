\documentclass[11pt,english,a4paper]{article}
\usepackage{babel}
\input{/home/marius/Dokumenter/preamples/phys_en.pre}
\author{\normalsize Marius Jonsson (Institutt for Vanskelig Fysikk, Oscars gate 19, 0352 OSLO, Norway) \\\\
\vspace{5px}
\normalsize \texttt{http://github.com/kingoslo/flintstones}}
\title{\bf \uppercase{Some title}}
\date{\normalsize \today}
\addbibresource{/home/marius/Dokumenter/MyLibrary.bib}
\DeclareUnicodeCharacter{2212}{$-$}
\begin{document}
\maketitle
\begin{abstract} \normalsize This is a report submission for the first project of «Computational physics 2» at the Institute of Physics, University of Oslo, autumn 2016.
\end{abstract}
\lstset{
  xleftmargin=.2\textwidth, xrightmargin=.2\textwidth
}

\section*{\uppercase{Introduction}}
A.\\
\\
The report is structured by «introduction»-, «methods»-, «results and discussion»- and finally a «conclusion and perspectives»-sections.
\section*{\uppercase{Methods}}
Suppose $\floor{\cdot}$ denotes the floor function on $\mathbb{R}$, then we know that the Hermite polynomials are given by
\[
H_n(x) = n! \sum_{m=0}^{\floor{n/2}} \frac{(-1)^m}{m!(n-2m)!}(2x)^{n-2m}.
\]
\texttt{(nx,ny,spin,energy)}:
\lstinputlisting{../resources/output.txt}
Suppose the principal quantum numbers of two-electron state indexed by $t$ are $n_t,m_t$, $V(r) = -e^2/(4\pi^2\varepsilon_0 r)$ denote the Coloumb potential, $p,q,r,s \in \{1,2,\cdots\}$ and let $A=\{ p,q,r,s \}$, $B = \{(1,p),(2,q),(1,r),(2,s)\}$ then a parameterization of the matrix element $\langle pq | V  |  rs \rangle$ is
\begin{align*}
\langle pq | V  |  rs \rangle &= \iiiint_{\mathbb{R}^4} \psi^*_p(\vec{r}_1)\psi^*_q(\vec{r}_2)V(\| \vec{r}_1-\vec{r}_2 \|)\psi^*_r(\vec{r}_1)\psi^*_s(\vec{r}_2)\,\diff \vec{r}_1 \, \diff \vec{r}_2\\
&= - C \int_{-\infty}^\infty\int_{-\infty}^\infty\int_{-\infty}^\infty\int_{-\infty}^\infty \prod_{(k,t)\in B} H_{n_t}(u_k) H_{m_t}(v_k)\frac{\exp\Big(\sum_{i=1}^2 u_i^2 + v_i^2 \Big)\diff u_1 \diff v_1 \diff u_2 \diff v_2}{([u_1 - u_2]^2 + [v_1 - v_2]^2)^{1/2}}, \\
&\text{for normalization factor} \quad C = \frac{\omega^{1/2} e^2}{4 \pi^3 \varepsilon_0}\Big( 2^{\sum_{i \in A} n_i + m_i}\prod_{j \in A}n_j!m_j! \Big)^{  -1/2}.
\end{align*}
\[
n(\alpha) = \frac{1}{2} \left[ \left(\frac{E}{\hbar \omega} \right)(\alpha) - |m(\alpha)| - 1 \right]\]
\[ \quad m(\alpha) = - \left| \left(\frac{E}{\hbar \omega} \right)(\alpha) - 1 \right| + 2 \floor{ \frac{1}{2}\left[\alpha - 1 - \left( \left(\frac{E}{\hbar \omega} \right)(\alpha) - 1 \right)\left(\frac{E}{\hbar \omega} \right)(\alpha) \right] }
\]
\[
\left(\frac{E}{\hbar \omega} \right)(\alpha) = \ceil{\frac{1}{2}\left( 1 + 4\alpha \right)^{1/2} - \frac{1}{2} }
\]
\section*{\uppercase{Results and discussion}}

\section*{\uppercase{Conclusion and perspectives}}

\section*{\uppercase{Appendix}}
\begin{theorem}[Gaussian quadrature]
Suppose $A \subseteq \mathbb{R}$ and there exist an orthogonal basis $\{H_n\}_{n=0}^\infty$ of polynomials for the set of square integrable functions on $A$ with respect to the inner product
\[
\langle f,g\rangle = \int_A (Wfg)(x)\, \diff x.
\]
Suppose also that $H_n$ is a degree $n$-polynomial and $|\langle H_n,H_n \rangle| = c_n$. If $f: \mathbb{R} \to \mathbb{R}$ is integrable on $A$ and there exist and $N \in \mathbb{N}$ such that $f(x) = (WP_{2N-1})(x)$, then 
\[
\int_A f(x)\, \diff x = c_0 \sum_{i=1}^N (H^{-1})_{0n} P_{2N-1}(x_n),
\]
where $\{x_n\}_{n=1}^N$ are the zeros of $H_N$ and $(H^{-1})_{0n}$ is the inverse of the matrix with elements $H_{nk} = H_k(x_n)$.
\end{theorem}
\begin{proof}
Assume that the hypothesis is true, then in particular
\begin{equation}
f(x) = (WP_{2N-1})(x) \label{eq:product}
\end{equation}
Since $\{H_n\}_{n=1}^\infty$ is a polynomial basis for the space of square integrable $\mathbb{R} \to \mathbb{R}$-functions, there exist polynomials $Q_{N-1}$ and $R_{N-1}$, such that
\begin{equation}
P_{2N-1}(x) = H_N(x) R_{N-1} (x) + Q_{N-1}(x) = H_N(x) \sum_{k=0}^{N-1}r_nH_k(x) + \sum_{k=0}^{N - 1}q_k H_{k}(x) \label{eq:polynomial}
\end{equation}
Moreover, since the basis is orthogornal with respect to the given inner product, there exist normalization $c_{m}$ such that
\begin{equation}
\langle H_n,H_m\rangle = \int_A W(x) H_n(x) H_m(x)\, \diff x = c_{m}\delta_{mn}. \label{eq:orthonormality}
\end{equation}
Therefore the integral of interest is
\begin{align}
\int_A f(x) \,\diff x &\stackrel{\ref{eq:product}}{=} \int_A W(x)P_{2N-1}(x) \,\diff x \stackrel{\eqref{eq:polynomial}}{=} \int_A W(x)\left[ H_N(x) \sum_{k=0}^{N-1}r_nH_k(x) + \sum_{k=0}^{N - 1}q_k H_{k}(x) \right] \,\diff x \nonumber \\
&\stackrel{\eqref{eq:orthonormality}}{=} 0 + \sum_{k=0}^{N - 1}\int_A W(x) q_k H_{k}(x) \,\diff x = \sum_{k=0}^{N - 1} q_k \int_A W(x) H_{k}(x)\cdot\undercbrace{1}_{=H_0} \,\diff x \stackrel{\eqref{eq:orthonormality}}{=} \sum_{k=0}^{N - 1} q_kc_{k} \delta_{k0} \nonumber \\
&=q_0c_{0} \label{eq:coeff}
\end{align}
Since $H_N(x)$ is a degree $N$ polynomial by assumption, $H_N$ has exactly $N$ zeros by the fundamental theorem of algebra \parencite[12]{forster_lectures_1991}. Therefore there exist a set $\{x_k\}_{k=1}^N$ such that $H_N(x_k)=0$ for all $1 \leq k \leq N$. Define now $c_n = Q_{N-1}(x_n)$ and observe that
\begin{equation}
c_n = Q_{N-1}(x_n) \stackrel{\eqref{eq:polynomial}}{=} \sum_{k=0}^{N-1}q_k H_k(x_n) \equiv \sum_{k=0}^{N-1}q_k H_{nk} \label{eq:cn}
\end{equation}
Since $\{H_n\}$ is a basis, each element is linearly independent, and therefore the matrix consisting of elements $H_{nk}$ is invertible with inverse $(H^{-1})_{mn}$. By solving for $b_k$ we obtain
\begin{equation}
\sum_{n=0}^{N-1}(H^{-1})_{mn}Q_{N-1}(x_n) = \sum_{n=0}^{N-1}(H^{-1})_{mn} c_n \stackrel{\eqref{eq:cn}}{=} \sum_{k=0}^{N-1}\sum_{n=0}^{N-1}q_k(H^{-1})_{mn} H_{nk} = \sum_{k=0}^{N-1}q_k \delta_{mk} = q_{m} \label{eq:qm}
\end{equation}
But since $\{x_k\}$ are the zeros ($\dagger$) of $H_N$, we see that
\begin{align}
q_m &\stackrel{\eqref{eq:qm}}{=} \sum_{n=0}^{N-1}(H^{-1})_{mn}Q_{N-1}(x_n) \stackrel{\eqref{eq:polynomial}}{=} \sum_{n=0}^{N-1}(H^{-1})_{mn}\Big[ P_{2N-1}(x_n) - \undercbrace{H_N(x_n)}_{=0 \quad (\dagger) } R_{N-1} (x_n) \Big] \nonumber\\
& \stackrel{(\dagger)}{=} \sum_{n=0}^{N-1}(H^{-1})_{mn}P_{2N-1}(x_n) \qquad \text{only if}\qquad q_0 = \sum_{n=0}^{N-1}(H^{-1})_{0n}P_{2N-1}(x_n). \label{q0}
\end{align}
By setting \eqref{q0} equal to \eqref{eq:coeff}, the theorem follows.
\end{proof}

\printbibliography
\end{document}